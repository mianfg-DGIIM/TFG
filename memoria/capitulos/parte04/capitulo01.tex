% !TeX root = ../libro.tex
% !TeX encoding = utf8

\setchapterpreamble[c][0.75\linewidth]{%
	\sffamily
  \emph{La conclusión ineludible parece ser que los matemáticos no utilizan un procedimiento de cálculo asumidamente sólido para determinar la verdad matemática. Deducimos que la comprensión matemática --el medio por el que los matemáticos llegan a sus conclusiones con respecto a la verdad matemática-- ¡no puede reducirse al cálculo ciego!}
 \begin{flushright} — Roger Penrose, \cite{Penrose1994} \end{flushright}
% \\[8pt]
	\par\bigskip
}
\vspace{28pt}

\chapter{Conclusiones}\label{ch:conclusiones}

Llegado al final de este trabajo, podemos decir que hemos cumplido satisfactoriamente nuestros objetivos. En este capítulo, resumiremos los logros alcanzados, y destacaremos la relevancia e impacto de esta obra.

Hemos logrado probar el Primer Teorema de Incompletitud desde una perspectiva computacional [T1]. El resultado obtenido no es exactamente el de Gödel, pero es de excepcional relevancia. Además, también obtenemos un resultado semántico de gran interés. Adicionalmente, probamos los teoremas (así como todos los resultados necesarios) con la formalidad que merecen. Esto nos permite una comprensión más profunda de la relación entre la computabilidad y la demostrabilidad.

Demostramos la mayoría de los resultados usando programas en Python [T2]. Aunque debemos dedicar una sección completa a probar que esto es riguroso, este desarrollo teórico nos permite usar un lenguaje de programación para nuestras demostraciones, mucho más intuitivo y comprensible para personas sin conocimientos de teoría de la computación. Esto nos permite crear una librería ejecutable de programas que guían las diversas demostraciones del trabajo, facilitando el entendimiento de los conceptos abordados [P2].

También hemos expuesto las consecuencias del Primer Teorema de Incompletitud en las disciplinas matemática y filosófica [T3], así como la relación con la tesis de Church-Turing [T5]. Por otra parte, hemos contextualizado adecuadamente el objeto de este trabajo en el ámbito histórico [T4], reconociendo la relevancia e impacto en las matemáticas y la filosofía. Este análisis histórico ha proporcionado una base sólida para la comprensión y la apreciación de los resultados de este trabajo.

Por otra parte, hemos de destacar la rigurosidad del trabajo realizado, precisando los conceptos relacionados con la teoría de la demostración [T6]. En concreto, se ha pretendido distinguir entre los conceptos de verdad y demostración, introduciendo múltiples ejemplos para ilustrar las diferencias entre los aspectos sintácticos y semánticos del razonamiento matemático.

Además de los logros académicos, este trabajo tiene un valor adicional como herramienta de apoyo a la docencia [P1]. El contenido creado puede ser usado para facilitar la enseñanza y el aprendizaje de la incompletitud desde una perspectiva computacional, ofreciendo ejemplos claros y programas en Python que ayudarán a los estudiantes a comprender y aplicar los temas tratados de manera práctica.

\endinput
